% This is the Reed College LaTeX thesis template. Most of the work 
% for the document class was done by Sam Noble (SN), as well as this
% template. Later comments etc. by Ben Salzberg (BTS). Additional
% restructuring and APA support by Jess Youngberg (JY).
% Your comments and suggestions are more than welcome; please email
% them to cus@reed.edu
%
% See http://web.reed.edu/cis/help/latex.html for help. There are a 
% great bunch of help pages there, with notes on
% getting started, bibtex, etc. Go there and read it if you're not
% already familiar with LaTeX.
%
% Any line that starts with a percent symbol is a comment. 
% They won't show up in the document, and are useful for notes 
% to yourself and explaining commands. 
% Commenting also removes a line from the document; 
% very handy for troubleshooting problems. -BTS

% As far as I know, this follows the requirements laid out in 
% the 2002-2003 Senior Handbook. Ask a librarian to check the 
% document before binding. -SN

%%
%% Preamble
%%
% \documentclass{<something>} must begin each LaTeX document
\documentclass[12pt,twoside]{reedthesis}
% Packages are extensions to the basic LaTeX functions. Whatever you
% want to typeset, there is probably a package out there for it.
% Chemistry (chemtex), screenplays, you name it.
% Check out CTAN to see: http://www.ctan.org/
%%
\usepackage{graphicx,latexsym} 
\usepackage{amssymb,amsthm,amsmath}
\usepackage{longtable,booktabs,setspace} 
\usepackage{chemarr} %% Useful for one reaction arrow, useless if you're not a chem major
\usepackage[hyphens]{url}
\usepackage{rotating}
\usepackage{natbib}
% Comment out the natbib line above and uncomment the following two lines to use the new 
% biblatex-chicago style, for Chicago A. Also make some changes at the end where the 
% bibliography is included. 
%\usepackage{biblatex-chicago}
%\bibliography{thesis}

\newtheorem{definition}{Definition}
% \usepackage{times} % other fonts are available like times, bookman, charter, palatino

\title{Multi-Input Functional Encryption and Obfuscation}
\author{Sage R. Michaels}
% The month and year that you submit your FINAL draft TO THE LIBRARY (May or December)
\date{May 2018}
\division{Mathematics and Natural Sciences}
\advisor{Dylan McNamee}
%If you have two advisors for some reason, you can use the following
%\altadvisor{Your Other Advisor}
%%% Remember to use the correct department!
\department{Mathematics}
% if you're writing a thesis in an interdisciplinary major,
% uncomment the line below and change the text as appropriate.
% check the Senior Handbook if unsure.
\thedivisionof{The Established Interdisciplinary Committee for}
% if you want the approval page to say "Approved for the Committee",
% uncomment the next line
%\approvedforthe{Committee}

\setlength{\parskip}{0pt}
%%
%% End Preamble
%%
%% The fun begins:
\begin{document}

  \maketitle
  \frontmatter % this stuff will be roman-numbered
  \pagestyle{empty} % this removes page numbers from the frontmatter

% Acknowledgements (Acceptable American spelling) are optional
% So are Acknowledgments (proper English spelling)
    \chapter*{Acknowledgements}
	I want to thank a few people.

% The preface is optional
% To remove it, comment it out or delete it.
    \chapter*{Abstract}
	This is an example of a thesis setup to use the reed thesis document class.
	

    \tableofcontents
% if you want a list of tables, optional
%    \listoftables
% if you want a list of figures, also optional
 %   \listoffigures
    
 
  \mainmatter % here the regular arabic numbering starts
  \pagestyle{fancyplain} % turns page numbering back on   
    
     \chapter*{Introduction}
         \addcontentsline{toc}{chapter}{Introduction}
	\chaptermark{Introduction}
	\markboth{Introduction}{Introduction}
	        
    
    \chapter{Background}
    \section{Encryption}
    In plain English, Encryption a way to share a message so that only the intended recipient(s) of
    that message are able to read it. Historically this was done by means of obscurity, in the sense that correspondents
    assumed only they knew the specific method by which messages between them would be encrypted. The problem
    with Encryption by obscurity is that as soon as a method of obscurity becomes popular, it immediately becomes
    obsolete.
    \subsection{Classical Encryption}
    Now, Cryptographers work to develop encryption schemes that are computationally infeasible for adversaries to break even if the method of encryption is known (this is known as Kerckhoff's Principle). To do this, Encryption functions are made public but take an extra parameter that is kept secret, we call this secret a key, and the best keys are ones that are chosen randomly, since they are nearly impossible to guess. In defining an encryption scheme it is important to note that there exists a keyspace $K$ which is the set of all valid keys, a message space $M$ made up of all valid messages, and a cipher space $C$ the set of all valid cipher-texts (encryptions of messages).
 \\ \\
  We define an encryption scheme $\Pi$ to be the following three functions:
     
 
 $$\text{Gen}:\mathbb{Z} \rightarrow K \times K$$
 Defined to be for $\lambda \in \mathbb{Z}$, Gen($\lambda ) \rightarrow (pk,sk)$ where $pk$ and $sk$ are seemingly random keys of length $\lambda$.
 
  $$\text{Enc}:K \times M \rightarrow C$$
Defined to be for key $pk\in K$ and message $m\in M$ Enc$_pk(m) \rightarrow c$ for some cipher text $c\in C$
 
 $$\text{Dec}:K \times C \rightarrow M$$
 Defined to be for key $sk \in K$ and cipher text $c\in C$ Dec$_sk(c) \rightarrow m$ for some message $m\in M$
\par It is important to note that if $sk = pk$ this is called a symmetric or private key encryption scheme meaning only the correspondents know the key and they keep it secret. If $sk \not= pk$ then this is called an asymmetric or public key encryption scheme where $sk$ is a secret key and $pk$ is a public key. In a public key encryption scheme anyone can encrypt a message since the public key is public, but only people with the secret key are able to decrypt.


\begin{definition}[Correctness]
In this setting we say an encryption scheme $\Pi$ is \textbf{correct} if for $n\in \mathbb{Z} , (sk,pk) \leftarrow$ Gen$(n)$ and $m\in M$ 

$$\text{Dec}_{sk}(\text{Enc}_{pk}(m)) = m$$
\end{definition}


\par Suppose Alice and wants to send Bob a secret message $m$. To do this Bob would have to run Gen$(n) \rightarrow pk, sk$ and then send $pk$ to Alice. Then Alice gets $c:= $Enc$_{pk}(m)$ and sends $c$ over to Bob. Finally Bob gets $m':= $Dec$_{sk}(c)$. If the scheme is correct then $m' = m$ and Bob is able to read Alice's message. The above interaction is represented in the following diagram.
\begin{center}
\textbf{insert sick diagram of Alice interacting with Bob}
\end{center}


\newcommand{\Enc}[0]{Enc}
\newcommand{\Dec}[0]{Dec}
\subsection{Functional Encryption}

With classical encryption, decryption is all or nothing, either you have the secret key and can find out the message, or you don't have the secret key so you can't. With functional encryption we broaden the possibilities of what is communicated between senders in an encryption scheme. We start with a definition and then show the formal construction.
\begin{definition}[Correctness]
A Functional Encryption Scheme $\Pi$ is \textbf{correct} if for $m \in M$, some predetermined function $f$ with $M$ as it's domain, and appropriate $(pk,ek)\in K$ generated by $\Pi$'s key generation algorithm:

$$\Dec_{ek}(\Enc_{pk}(m)) = f(m) $$
\end{definition}

It's easy to see that this definition encapsulates the older definition of correctness by making $f$ the identity function $f(m) = m$, but this syntax covers many other cryptographic primitives as well like Attribute Based Encryption and Identity Based Encryption. To see how these primitives are sub cases of Functional Encryption. Lets formalize our idea of a Functional Encryption Scheme. 
\par To define a Functional Encryption Scheme, we must first define a way of describing what a cipher text can be decrypted to.

\begin{center}
\textbf{Think of something better than case space, it's confusing with the notation for a cipher space. The paper calls it a key space K but that's also confusing. Change later, keep in mind now.}
\end{center}


\begin{definition}[Functionality]
Given a case space $C_ase \cup \{\epsilon\}$, message space $M$ we define the functionality $F$ to be 
$$F:C_ase \times M \rightarrow M$$
\end{definition}

\par Functionality describes what the possible outputs are. In public key encryption, knowing the secret key $sk$ allows for the message to be read in full, but without the secret key, only the length of the message can be discerned from the cipher text. To write this in the syntax of a functionality we define
$$
F(c,m) = 
\begin{cases}
 x \text{ if } c = 1 \\
\text{length}(x) \text{ if } c = \epsilon
\end{cases}
$$
The only functionality of public key encryption is fully decoding the message so this is our primary case $(c = 1)$, however we also account for the information learned without the public key which is an unavoidable rather than built in case $(c = \epsilon)$.


\newcommand{\Z}[0]{\mathbb{Z}}
\begin{definition}[Functional Encryption Scheme]
A Functional Encryption Scheme $\Pi$ is defined to be the following algorithms:

$$\text{Setup}: \Z \rightarrow K \times K$$
Defined: For $\lambda \in \Z,$ setup($\lambda) \rightarrow (pk,mk)$, generates a public key and master key
$$\text{Gen}: K \times C \rightarrow K $$
Defined: For $c\in C_{ase}, mk \in K$, Gen$(mk,c) \rightarrow sk$ which is kept secret and is the secret key of functionality $c$.

$$\Enc: K \times M \rightarrow C_{ipher}$$
Defined: For $pk\in K$ and $m \in M, \Enc_{pk}(m) \rightarrow c$

$$\Dec:K \times C_{ipher} \rightarrow M$$
Defined: For $ek \in K$ and $c \in C_{ipher}, \Dec(k,c) \rightarrow n$ where $n = F(k,m)$ for some functionality $F$. 
\end{definition}


If the notion of a Functionality was confusing before, the use of it in generating the secret key should make it clear.

\par Suppose Alice wants to store some text files ($\{m_1,m_2,\cdots, m_n \}$) on Bob's Cloud Storage to retrieve at some later date.
 Bob, owner of Bob's Cloud Storage, has a strong dislike for cats and doesn't want anything stored on
  his cloud with the word "cat" in it. Alice and Bob contact Janet, impartial third party, and share their desired functionalities: Alice wants to be able to fully decrypt (i.e. identity functionality), Bob wants to be able to check if the word "cat" is in anything stored on his server (we call this function checking for the word "cat" $f$). Janet runs Setup$(\lambda)\rightarrow (mk, pk)$ and publishes the public key, and also runs Keygen$(mk,0) \rightarrow sk_{id}$ and Keygen$(mk,1) \rightarrow sk_f$ for functionality:
$$F(c,m) = \begin{cases}
f(m) \text{ if } k = 0 \\
id(m) \text{ if } k = 1
\end{cases} $$
Then Janet securely sends $sk_{id}$ to Alice and $sk_{f}$ to Bob. Now Alice encrypts her text files Enc$_{pk}(m_1) \rightarrow c_1,$Enc$_{pk}(m_2 \rightarrow c_2, \cdots $,Enc$_{pk}(m_n) \rightarrow c_n,$, and sends $\{c_1,c_2, \cdots, c_n \}$ to Bob for storage. Bob then runs $\Dec_{sk_f}(c_1) \rightarrow b_1 , \cdots, \Dec_{sk_f}(c_n) \rightarrow b_n $ and allows storage of all $c_i$ that indicate the word "cat" is not being stored.  This interaction is illustrated in the diagram below:

\begin{center}

\textbf{ insert sick diagram of Alice Bob and Janet executing the exchange discussed above!}
\end{center}

\par It should be noted that the impartial 3rd party does not need to be fully trusted since Janet doesn't have any of Alice's cipher text's to decrypt, but it is important that Janet is impartial since otherwise should could collaborate with Alice to store a Cat Encyclopedia on Bob's Servers or could collaborate with Bob to learn more about what Alice is storing than what Alice and Bob have agreed upon.


\par Functional Encryption is at the early stages of development now, but is an extremely powerful tool. From Functional Encryption we can easily describe variations like Attribute Based Encryption, Identity Based Encryption, and Multi Input Functional Encryption.


    \section{Obfuscation}
    
    \subsection{Black Box Obfuscation}
    In short, circuit (and program) Obfuscation is a Cryptographic method that seeks to make a circuit
    unintelligible to anyone looking at only its obfuscation. Black Box Obfuscation was first defined by
    \textbf{(INSERT CITATION)}. The goal of Black Box Obfuscation is that a circuit that has been Black Box Obfuscated should reveal no more about it's inner workings than a table of inputs and outputs of the that circuit (we call this table an Oracle since).
    
    \begin{definition}
    Given any circuit $C$ and an Oracle $\mathcal{O}_c(\cdot)$ with the same functionality of $C$, an obfuscater $\mathcal{O}(\cdot)$, and any adversary $\mathcal{A}$. We say $\mathcal{O}$ is a Black Box obfuscater if $\mathcal{A}$ given access to $\mathcal{O}(C)(\cdot)$ can determine just as much about the inner workings of $C$ as if $\mathcal{A}$ had been given access to $\mathcal{O}_c(\cdot)$.
   
    
    \end{definition}
    
    \begin{center}
    \textbf{ FIND A BETTER DEFINITION OF BLACK BOX OBFUSCATION}
    \end{center}
    
    \par While Black Box Obfuscation has been proven impossible for general circuits and programs (\textbf{insert citation and perhaps some more details as to what exactly has been proven impossible}) Let's take a moment to talk about why Cryptographers would want Black Box Obfuscation as a tool. Currently Public Key Encryption relies on expensive computations, since the private keys are just 
    inverses of public keys in some group so the groups and the keys have to be extremely large, however private key encryption is much more efficient since it's just running some sort of permutation on the message using the key and then descrambling the cipher text with the same key. Using Black 
    Box Obfuscation if would be possible to obfuscate an encryption function with a secret key, $sk$, baked in $\mathcal{O}(\Enc_{sk}(\cdot) \rightarrow \Enc(\cdot)$ where $\Enc(\cdot)$ can be made public without risk of anyone learning $sk$ keeping the ability to decrypt in the hands of those who started off with the secret key.
    
    \subsection{Indistinguishability Obfuscation}
    While Black Box Obfuscation was proved impossible, Cryptographers weakened their definition of Obfuscation to try and see what \textit{is} possible in the field of obfuscation. This led to a new notion of Obfuscation called Indistinguishability Obfuscation. 
    \newcommand{\iO}[0]{\textit{i}\mathcal{O}}
    
    
    \begin{definition}[Indistinguishability Obfuscation]
    Given circuits $C_0,\ C_1$ where $|C_0| \approx |C_1|, C_0(x) = C_1(x)$ for all valid $x$, and an obfuscater $\iO(\cdot)$; we say that $\iO$ is an \textbf{indistinguishability obfuscater} if for all Distinguishers $\mathcal{D}$
    $$\Pr[\mathcal{D}(\iO(C_0)) = 1] -\Pr[\mathcal{D}(\iO(C_1)) = 1] \leq \text{negl}$$
    \end{definition}
    
    \par This definition can be a little confusing to understand. In short, Indistinguishability Obfuscation guarantees that two circuits with the same functionality are indistinguishable from one another once run through an indistinguishability obfuscater. 
    \par What can be more confusing is why this would be useful since the two circuits are functionally the same. The most obvious usage is in removing water marks from programs. Suppose Dan buys a fancy piece of software called Macrosoft Word for his company. Macrosoft is worried about their software being pirated so they put a watermark in Dan's copy of Macrosoft Word that doesn't change the functionality of his copy of the program. Suppose Dan wants to make Macrosoft Word free for everyone and decides to post it on a torrenting website. If Dan posts his copy as is, Macrosoft and their Copyright lawyers will be able to see that it was Dan who illegally shared his software. Instead, if Dan first runs his copy of Macrosoft word through an Indistinguishability Obfuscater and post that, Macrosoft and their lawyers will be unable to tell if it was Dan's copy that was posted, or another customer's.
    \par This might seem like a weak definition, but Indistinguishability Obfuscation has become a powerful cryptographic tool, and has proven to be as good as the best possible obfuscation, the proof is short and eloquent so we will show it here but it was originally formulated by (\textbf{INSERT CITATION HERE})
    
    \newcommand{\Oh}[0]{\mathcal{O}}
    \begin{proof}
    Let $\iO(\cdot)$ be an indistinguishability obfuscater, $\Oh(\cdot)$ the best obfuscater possible, and $C$ a circuit. Then by definition of Indistinguishability Obfuscation, $\iO(C)$ is indistinguishable from $\Oh(C)$. Thus Indistinguishability Obfuscation is at least as good as any other type of Obfuscation.
    \end{proof}
    
    Because of this, Cryptographers often treat Indistinguishability Obfuscation the same as Black Box Obfuscation. This could lead to issues in the future and there is lots of room for work to be done in quantifying levels of Obfuscation.
    
    
    
    
    \chapter{Multi-Linear Maps}
    \section{Definition}
    
    \section{Intuition}
    \section{Construction Outline}
    \section{Candidate Goups/Quotient Rings/Fields}
    
    
    
    \chapter{Indistinguishability Obfuscation}
    \section{Construction}
    \section{Usage, Limitations, and Goals}
    
    
    
    
    \chapter{Multi-Party Input Functional Encryption}
    \section{Scheme}
    \section{Construction}
    \section{Limitations and Goals}
    
    
    
    \chapter{ A Brief Introduction to the 5-GenC library}
    \section{ The DSL}
    \section{Circuits and Branching Programs}
    \section{Base and MMaps}
    
    \chapter{Experiments}
    \section{Comparison Circuit}
    \section{Runtime Evaluation}
    
    
    \chapter{Conclusion} 
    
    
    
    

 \bibliographystyle{APA/apa-good}  % or
 \bibliography{thesis}
 % Comment the above two lines and uncomment the next line to use biblatex-chicago.
 %\printbibliography[heading=bibintoc]

% Finally, an index would go here... but it is also optional.
\end{document}
