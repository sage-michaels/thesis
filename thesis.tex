% This is the Reed College LaTeX thesis template. Most of the work 
% for the document class was done by Sam Noble (SN), as well as this
% template. Later comments etc. by Ben Salzberg (BTS). Additional
% restructuring and APA support by Jess Youngberg (JY).
% Your comments and suggestions are more than welcome; please email
% them to cus@reed.edu
%
% See http://web.reed.edu/cis/help/latex.html for help. There are a 
% great bunch of help pages there, with notes on
% getting started, bibtex, etc. Go there and read it if you're not
% already familiar with LaTeX.
%
% Any line that starts with a percent symbol is a comment. 
% They won't show up in the document, and are useful for notes 
% to yourself and explaining commands. 
% Commenting also removes a line from the document; 
% very handy for troubleshooting problems. -BTS

% As far as I know, this follows the requirements laid out in 
% the 2002-2003 Senior Handbook. Ask a librarian to check the 
% document before binding. -SN

%%
%% Preamble
%%
% \documentclass{<something>} must begin each LaTeX document
\documentclass[12pt,twoside]{reedthesis}
% Packages are extensions to the basic LaTeX functions. Whatever you
% want to typeset, there is probably a package out there for it.
% Chemistry (chemtex), screenplays, you name it.
% Check out CTAN to see: http://www.ctan.org/
%%
\usepackage{graphicx,latexsym} 
\usepackage{amssymb,amsthm,amsmath}
\usepackage{longtable,booktabs,setspace} 
\usepackage{chemarr} %% Useful for one reaction arrow, useless if you're not a chem major
\usepackage[hyphens]{url}
\usepackage{rotating}
\usepackage{natbib}
% Comment out the natbib line above and uncomment the following two lines to use the new 
% biblatex-chicago style, for Chicago A. Also make some changes at the end where the 
% bibliography is included. 
%\usepackage{biblatex-chicago}
%\bibliography{thesis}

% \usepackage{times} % other fonts are available like times, bookman, charter, palatino

\title{Multi-Input Functional Encryption and Obfuscation}
\author{Sage R. Michaels}
% The month and year that you submit your FINAL draft TO THE LIBRARY (May or December)
\date{May 2018}
\division{Mathematics and Natural Sciences}
\advisor{Dylan McNamee}
%If you have two advisors for some reason, you can use the following
%\altadvisor{Your Other Advisor}
%%% Remember to use the correct department!
\department{Mathematics}
% if you're writing a thesis in an interdisciplinary major,
% uncomment the line below and change the text as appropriate.
% check the Senior Handbook if unsure.
\thedivisionof{The Established Interdisciplinary Committee for}
% if you want the approval page to say "Approved for the Committee",
% uncomment the next line
%\approvedforthe{Committee}

\setlength{\parskip}{0pt}
%%
%% End Preamble
%%
%% The fun begins:
\begin{document}

  \maketitle
  \frontmatter % this stuff will be roman-numbered
  \pagestyle{empty} % this removes page numbers from the frontmatter

% Acknowledgements (Acceptable American spelling) are optional
% So are Acknowledgments (proper English spelling)
    \chapter*{Acknowledgements}
	I want to thank a few people.

% The preface is optional
% To remove it, comment it out or delete it.
    \chapter*{Abstract}
	This is an example of a thesis setup to use the reed thesis document class.
	

    \tableofcontents
% if you want a list of tables, optional
%    \listoftables
% if you want a list of figures, also optional
 %   \listoffigures
    
 
  \mainmatter % here the regular arabic numbering starts
  \pagestyle{fancyplain} % turns page numbering back on   
    
     \chapter*{Introduction}
         \addcontentsline{toc}{chapter}{Introduction}
	\chaptermark{Introduction}
	\markboth{Introduction}{Introduction}
	        
    
    \chapter{Background}
    \section{Encryption}
       In plain English, Encryption is any way to share a message so that only the intended recipient(s) of
    that message are able to read it. Historically this was done by means of obscurity, in the sense that correspondents
    assumed only they knew the specific method by which messages between them would be encrypted. The problem
    with Encryption by obscurity is that as soon as a method of obscurity becomes popular, it immediately becomes
    obsolete.
    \subsection{Modern Encryption}
    Now, Cryptographers' work to develop encryption schemes that are computationally infeasible for adversaries to break even if the method of encryption is known (this is known as Kerckhoff's Principle). Suppose Alice wishes to send a private message to Bob, 
     
    
      
    \section{Black Box Obfuscation}
    \section{Diffie-Hellman Key Exchange}
    
    \chapter{Multi-Linear Maps}
    \section{Definition}
    \section{Intuition}
    \section{Construction Outline}
    \section{Candidate Goups/Quotient Rings/Fields}
    
    
    
    \chapter{Indistinguishability Obfuscation}
    \section{Definition}
    \section{Construction}
    \section{Usage, Limitations, and Goals}
    
    
    
    
    \chapter{Multi-Party Input Functional Encryption}
    \section{Scheme}
    \section{Construction}
    \section{Limitations and Goals}
    
    
    
    \chapter{ A Brief Introduction to the 5-GenC library}
    \section{ The DSL}
    \section{Circuits and Branching Programs}
    \section{Base and MMaps}
    
    \chapter{Experiments}
    \section{Comparison Circuit}
    \section{Runtime Evaluation}
    
    
    \chapter{Conclusion} 
    
    
    
    

 \bibliographystyle{APA/apa-good}  % or
 \bibliography{thesis}
 % Comment the above two lines and uncomment the next line to use biblatex-chicago.
 %\printbibliography[heading=bibintoc]

% Finally, an index would go here... but it is also optional.
\end{document}
