% This is the Reed College LaTeX thesis template. Most of the work 
% for the document class was done by Sam Noble (SN), as well as this
% template. Later comments etc. by Ben Salzberg (BTS). Additional
% restructuring and APA support by Jess Youngberg (JY).
% Your comments and suggestions are more than welcome; please email
% them to cus@reed.edu
%
% See http://web.reed.edu/cis/help/latex.html for help. There are a 
% great bunch of help pages there, with notes on
% getting started, bibtex, etc. Go there and read it if you're not
% already familiar with LaTeX.
%
% Any line that starts with a percent symbol is a comment. 
% They won't show up in the document, and are useful for notes 
% to yourself and explaining commands. 
% Commenting also removes a line from the document; 
% very handy for troubleshooting problems. -BTS

% As far as I know, this follows the requirements laid out in 
% the 2002-2003 Senior Handbook. Ask a librarian to check the 
% document before binding. -SN

%%
%% Preamble
%%
% \documentclass{<something>} must begin each LaTeX document
\documentclass[12pt,twoside]{reedthesis}
% Packages are extensions to the basic LaTeX functions. Whatever you
% want to typeset, there is probably a package out there for it.
% Chemistry (chemtex), screenplays, you name it.
% Check out CTAN to see: http://www.ctan.org/
%%
\usepackage{graphicx,latexsym} 
\usepackage{amssymb,amsthm,amsmath}
\usepackage{longtable,booktabs,setspace}
\usepackage{tikz}
\usetikzlibrary{tikzmark,calc}
\usepackage{tikz-cd}
\usepackage{chemarr} %% Useful for one reaction arrow, useless if you're not a chem major
\usepackage[hyphens]{url}
\usepackage{rotating}
\usepackage{natbib}
% Comment out the natbib line above and uncomment the following two lines to use the new 
% biblatex-chicago style, for Chicago A. Also make some changes at the end where the 
% bibliography is included. 
%\usepackage{biblatex-chicago}
%\bibliography{thesis}

\newtheorem{definition}{Definition}
\newtheorem{theorem}{Theorem}
\newtheorem{assumption}{Hardness Assumption}
% \usepackage{times} % other fonts are available like times, bookman, charter, palatino

\title{Multi-Input Functional Encryption and Obfuscation}
\author{Sage R. Michaels}
% The month and year that you submit your FINAL draft TO THE LIBRARY (May or December)
\date{May 2018}
\division{Mathematics and Natural Sciences}
\advisor{Dylan McNamee}
%If you have two advisors for some reason, you can use the following
%\altadvisor{Your Other Advisor}
%%% Remember to use the correct department!
\department{Mathematics}
% if you're writing a thesis in an interdisciplinary major,
% uncomment the line below and change the text as appropriate.
% check the Senior Handbook if unsure.
\thedivisionof{The Established Interdisciplinary Committee for}
% if you want the approval page to say "Approved for the Committee",
% uncomment the next line
%\approvedforthe{Committee}

\newcommand{\coloneqq}[0]{:=}
\newcommand{\gen}[0]{\text{Gen}}
\newcommand{\enc}[0]{\text{Enc}}
\newcommand{\dec}[0]{\text{Dec}}
\newcommand{\Z}[0]{\mathbb{Z}}

\setlength{\parskip}{0pt}
%%
%% End Preamble
%%
%% The fun begins:
\begin{document}

  \maketitle
  \frontmatter % this stuff will be roman-numbered
  \pagestyle{empty} % this removes page numbers from the frontmatter

% Acknowledgements (Acceptable American spelling) are optional
% So are Acknowledgments (proper English spelling)
    \chapter*{Acknowledgements}
	I want to thank a few people.

% The preface is optional
% To remove it, comment it out or delete it.
    \chapter*{Abstract}
	This is an example of a thesis setup to use the reed thesis document class.
	

    \tableofcontents
% if you want a list of tables, optional
%    \listoftables
% if you want a list of figures, also optional
 %   \listoffigures
    
 
  \mainmatter % here the regular arabic numbering starts
  \pagestyle{fancyplain} % turns page numbering back on   
    
     \chapter*{Introduction}
         \addcontentsline{toc}{chapter}{Introduction}
	\chaptermark{Introduction}
	\markboth{Introduction}{Introduction}
	        
    
    \chapter{Background}
    \section{Encryption}
    Encryption is a way to share a message so that only the intended recipient(s) of
    that message are able to read it. Historically this was done by means of obscurity, in the sense that correspondents assumed only they knew the specific method by which messages between them would be encrypted. The problem with encryption via obscurity is that as soon as a method of encryption becomes popular, it also becomes obsolete.
    \subsection{Classical Encryption}
    Now, cryptographers work to develop encryption schemes that are computationally infeasible for adversaries to break even if the method of encryption is known (this is known as Kerckhoff's Principle). To do this, encryption functions are made public but take an extra parameter that is kept secret, we call this secret a key. The best keys are ones that are decently long and chosen randomly over some distribution (often uniformly), so it is infeasible for an adversary to check every possible key or guess the right key using information about communicating parties.
       \par In defining an encryption scheme we call the set of all valid keys $K$ called a key space, a set of all valid messages $M$ called the message space, and a set of all valid cipher texts (encrypted messages) $C$ called a cipher space.

\begin{definition}[Classical Encryption Scheme]
     
 
 $$\text{Gen}:\mathbb{Z} \rightarrow K \times K$$
 Defined to be for $\lambda \in \mathbb{Z}$, Gen($\lambda ) \rightarrow (pk,sk)$ where $pk$ and $sk$ are random keys of length $\lambda$.
 
  $$\text{Enc}:K \times M \rightarrow C$$
Defined for key $pk\in K$ and message $m\in M$; to be Enc$_{pk}(m) \rightarrow c$ for some cipher text $c\in C$
 
 $$\text{Dec}:K \times C \rightarrow M$$
 Defined for key $sk \in K$ and cipher text $c\in C$; to be Dec$_{sk}(c) \rightarrow m$ for some message $m\in M$
 
 \end{definition}
 
 
\par If $sk = pk$ this is called a symmetric or private key encryption scheme meaning only the correspondents know the key and they keep it secret. If $sk \not= pk$ then this is called an asymmetric or public key encryption scheme where $sk$ is a secret key and $pk$ is a public key. In a public key encryption scheme anyone can encrypt a message since the public key is public, but only people with the secret key are able to decrypt.


\begin{definition}[Correctness]
In this setting we say an encryption scheme $\Pi$ is \textbf{correct} if for $n\in \mathbb{Z} , (sk,pk) \leftarrow$ Gen$(n)$ and $m\in M$ 

$$\text{Dec}_{sk}(\text{Enc}_{pk}(m)) = m$$
\end{definition}


\par Suppose Alice and wants to send Bob a secret message $m$. To do this in the public key setting Bob would have to run Gen$(n) \rightarrow (pk, sk)$ and then send $pk$ to Alice. Then Alice gets $c:= $Enc$_{pk}(m)$ and sends $c$ over to Bob. Finally Bob gets $m':= $Dec$_{sk}(c)$. If the scheme is correct then $m' = m$ and Bob is able to read Alice's message. The above interaction is represented in the following diagram.


\begin{table}[ht]
  \centering
  \fbox{
  \begin{tabular}{c c c}
    Bob & & Alice \\ 
    $\gen(\lambda) \rightarrow (sk,pk)$ &&\\
    & $\xrightarrow{\quad pk \quad}$ & \\
    & & $\enc_{pk}(m) \rightarrow c$ \\
    & $\xleftarrow{\quad c \quad}$& \\
    $m' \coloneqq \dec_{pk}(c)$ && \\
  \end{tabular}
  }
   \caption{Public Key Encryption Protocol}
  \end{table}



\subsection{Functional Encryption}

With classical encryption, decryption is all or nothing, either you have the secret key and can find out the message, or you don't have the secret key so you can't. With functional encryption we broaden the possibilities of what is communicated between senders in an encryption scheme. We start with a definition and then show the formal construction.
\begin{definition}[Correctness]
A Functional Encryption Scheme $\Pi$ is \textbf{correct} if for $m \in M$, some predetermined function $f$ with $M$ as its domain, and appropriate public key and \textit{evaluation} key $(pk,ek)\in K$ from $\Pi$'s key generation algorithm:

$$\dec_{ek}(\enc_{pk}(m)) = f(m) $$
\end{definition}

It's easy to see that this definition encapsulates the older definition of correctness by making $f$ the identity function $f(m) = m$, but this syntax covers many other cryptographic primitives as well like attribute based encryption (ABE) and identity based encryption (IBE). To see how these primitives are sub cases of functional encryption lets formalize our idea of a functional encryption scheme. 




\begin{definition}[Functional Encryption Scheme]
Given a message space $M$, a cipher space $C$, a family of functions $\mathcal{F}$, and a key space $K$ , a functional encryption scheme $\Pi$ is defined to be the following four algorithms:

$$\text{Setup}: \Z \rightarrow K \times K$$
Defined for $\lambda \in \Z$; to be setup($\lambda) \rightarrow (PP,MSK)$, generates a public parameter and master secret key both in $K$
$$\text{Gen}: K \times \mathcal{F} \rightarrow K $$
Defined for $f \in \mathcal{F}, MSK \in K$; to be Gen$(mk,c) \rightarrow SK_f$ which is kept secret and is the secret key associated with the functionality of $f$.

$$\enc: K \times M \rightarrow C$$
Defined: For $PP\in K$ and $m \in M;$ to be $\enc_{PP}(m) \rightarrow c$

$$\dec:K \times C \rightarrow M$$
Defined for $SK_f \in K$ and $c \in C;$ to be $\dec(SK_f,c) \rightarrow m'$ if the scheme is correct then $m' = f(m)$ for some previously encrypted $m$ and some $f\in \mathcal{F}$.

\end{definition}



\par Suppose Alice wants to store some text files ($\{m_1,m_2,\cdots, m_n \}$) on Bob's Cloud Storage to retrieve at some later date.
 Bob, owner of Bob's Cloud Storage, has a strong dislike for cats and doesn't want anything stored on
  his cloud with the word ``cat'' in it. Alice and Bob contact Janet, an impartial third party, and share their desired functionalities: Alice wants to be able to fully decrypt (i.e. identity functionality), Bob wants to be able to check if the word ``cat'' is in anything stored on his server (we call this function checking for the word ``cat'' $f$). Janet runs Setup$(\lambda)\rightarrow (PP, MSK)$ and publishes the public parameter, and also runs Keygen$(MSK,id) \rightarrow SK_{id}$ and Keygen$(mk,f) \rightarrow SK_f$. Then Janet securely sends $sk_{id}$ to Alice and $sk_{f}$ to Bob. Now Alice encrypts her text files Enc$_{pk}(m_1) \rightarrow c_1,$Enc$_{pk}(m_2) \rightarrow c_2, \cdots $,Enc$_{pk}(m_n) \rightarrow c_n,$ and sends $\{c_1,c_2, \cdots, c_n \}$ to Bob for storage. Bob then runs $\dec_{sk_f}(c_1) \rightarrow b_1 , \cdots, \dec_{sk_f}(c_n) \rightarrow b_n $ and allows storage of all $c_i$ that indicate the word ``cat'' is not being stored.  This interaction is illustrated in the diagram below:
  
  

\begin{tikzcd}
\text{Bob} & \text{Janet} & \text{Alice} \\
{} & \text{Setup}(\lambda) \rightarrow (MSK,PP) \arrow[r, "PP"] \arrow[l, "PP"'] & {} \\
{} \arrow[r, "f"] & {} & {} \arrow[l, "\text{id}"'] \\
 & \text{KeyGen}(id,mk) \rightarrow SK_{\text{id}} \arrow[r, "\text{secure}(SK_{\text{id}})"] & {} \\
{} & \text{KeyGen}(f,mk) \rightarrow SK_{f} \arrow[l, "\text{secure}(SK_f)"'] &  \\
 &  & \text{Enc}_{PP}(m_1)\rightarrow c_1 \\
 &  & \vdots \\
 &  & \text{Enc}_{PP}(m_n)\rightarrow c_n \\
{} &  & {} \arrow[ll] \{c_1, \cdots, c_n \} \\
\text{Dec}_{SK_{f}}(c_1) \rightarrow b_1 &  &  \\
\vdots &  &  \\
\text{Dec}_{SK_{f}}(c_n) \rightarrow b_n &  & 
\end{tikzcd}


$\\$


\par It should be noted that the impartial 3rd party does not need to be fully trusted since Janet doesn't have any of Alice's cipher texts to decrypt, but it is important that Janet is impartial since otherwise should could collaborate with Alice to store a Cat Encyclopedia on Bob's Servers, or could collaborate with Bob to learn more about what Alice is storing than what Alice and Bob have agreed upon.


\par Research into functional encryption is currently at early stages, later we will describe the details of how it works, and where it's currently limited. Even at such an early stage cryptographic research, functional encryption has proven to be a powerful tool giving us a variety of functionaries with only small alterations of syntax such as attribute based encryption, identity based encryption , and multi-party input functional encryption [\cite{funcenc}] which we will actually work with later on. 

    
    

    \section{Obfuscation}
    
    \subsection{Black Box Obfuscation}
     \par Circuit (and program) obfuscation is a cryptographic method that seeks to make a circuit
    unintelligible to anyone looking at only its obfuscation. Black box obfuscation as defined in [\cite{vbb}] the goal of black box obfuscation is that a circuit that has been black box obfuscated should reveal no more about it's inner workings than a table of inputs and outputs for the same circuit (we call this table an Oracle).
    
    \begin{definition}
    Given any circuit $C$ and an Oracle $\mathcal{O}_c(\cdot)$ with the same functionality of $C$, an obfuscater $\mathcal{O}(\cdot)$, and any adversary $\mathcal{A}$. We say $\mathcal{O}$ is a Black Box obfuscater if $\mathcal{A}$ given access to $\mathcal{O}(C)(\cdot)$ can determine just as much about the inner workings of $C$ as if $\mathcal{A}$ had been given access to $\mathcal{O}_c(\cdot)$.    
    
    \end{definition}
        
     \par Let's take a moment to talk about why cryptographers would want black box obfuscation as a tool. Currently public key encryption relies on expensive computations, (for example in RSA the private keys are just inverses of public keys in some group large enough for this to be difficult to compute). However, private key encryption is much more efficient since it's just running some sort of permutation on the message using the secret key and then descrambling the cipher text with the same secret key. Using black box obfuscation if would be possible to obfuscate a private key encryption function with a secret key $sk$ baked in $\mathcal{O}(\enc_{sk}(\cdot) \rightarrow \enc(\cdot)$ where $\enc(\cdot)$ can be made public without risk of anyone learning $sk$ keeping the ability to decrypt in the hands of those who started off with the secret key. Thus Black Box Obfuscation would make efficient private key encryption into efficient public key encryption.
    
    \subsection{Indistinguishability Obfuscation}
    While Black Box Obfuscation was proved impossible in [\cite{vbb}], cryptographers weakened their definition of obfuscation to try and see what \textit{is} possible in the field of obfuscation. This led to a new notion of obfuscation called Indistinguishability Obfuscation. 
    \par Before defining indistinguishability obfuscation it is worth noting that a distinguisher is any polynomial time algorithm that can differentiate two things. For example, a distinguisher for apples and oranges might look at the color of what it has been given and output  Orange if the color is orange or Apple if the color of what it's been given is red/green/yellow, however another distinguisher for apples and oranges might just flip a coin and output Apple if the coin comes up heads and Orange otherwise, the second distinguisher would probably not be a very good distinguisher. Distinguishers are only defined on the two things that they distinguish, apples or oranges, green or blue, paper or not paper.
       \newcommand{\iO}[0]{\textit{i}\mathcal{O}}
    
    \pagebreak
    \begin{definition}{Indistinguishability Obfuscation}
    \par Given circuits $C_0,\ C_1$ where $|C_0| \approx |C_1|, C_0(x) = C_1(x)$ for all valid $x$, and an obfuscater $\iO(\cdot)$; we say that $\iO$ is an \textbf{indistinguishability obfuscater} if for all Distinguishers $\mathcal{D}$
    $$\Pr[\mathcal{D}(\iO(C_0)) = 1] -\Pr[\mathcal{D}(\iO(C_1)) = 1] \leq \text{negl}$$
    \end{definition}
    
    \par This definition can be a little confusing to understand. In short, Indistinguishability Obfuscation guarantees that two circuits with the same functionality are indistinguishable from one another once run through an indistinguishability obfuscater. 
    \par What can be more confusing is why this would be useful since we require the two circuits to be functionally the same. The most simple usage is in removing water marks from programs. Suppose Dan buys a fancy piece of software called Macrosoft Word for his company. Macrosoft is worried about their software being pirated so they put a watermark in Dan's copy of Macrosoft Word that doesn't change the functionality of his copy of the program, indicating that this is Dan's copy. Suppose Dan wants to make Macrosoft Word free for everyone and decides to post it on a torrenting website. If Dan posts his copy as is, Macrosoft and their copyright lawyers will be able to see that it was Dan who illegally shared his copy of their software. Instead, if Dan first runs his copy of Macrosoft word through an Indistinguishability Obfuscater and post that, Macrosoft and their lawyers will be unable to tell if it was Dan's copy that was posted, or another customer's.
    \par This might seem like a weak definition, but indistinguishability obfuscation has become a powerful cryptographic tool, and has proven to be the best possible obfuscation, the proof is short and eloquent so we will show it here but it was originally formulated by [\cite{Goldwasser:2007:BO:1760749.1760765}].
    
    \newcommand{\Oh}[0]{\mathcal{O}}
    \begin{proof}
    Let $\iO(\cdot)$ be an indistinguishability obfuscater, $\Oh(\cdot)$ the best obfuscater possible, and $C$ a circuit. Then by definition of Indistinguishability Obfuscation, $\iO(C)$ is indistinguishable from $\iO(\Oh(C))$. Thus Indistinguishability Obfuscation is at least as good as any other type of Obfuscation.
    \end{proof}
    
    Because of this, Cryptographers often treat Indistinguishability Obfuscation the same as Black Box Obfuscation. While this could lead to issues in the future and there is lots of room for work to be done in quantifying degrees of Obfuscation, we will fall back on the assumption that the best possible obfuscation is good enough for our purposes.
    
    
    \section{Bilinear Maps}
    
    Lastly we give a brief introduction to Bilinear maps, which we will later generalize to Multilinear maps.
    
    \subsection{Definition}
    
    \begin{definition}{Bilinear Maps}
    \par Let $G_1,G_2,G_3$ be cyclic groups of prime order $p$. Then we say a map $e:G_1 \times G_2 \rightarrow G_3$ is bilinear if
    
    \begin{enumerate}
    \item For all $g_1 \in G_1, g_2\in G_2,$ and $ \alpha \in \Z_p, \; e(\alpha\cdot g_1,g_2) =e( g_1,\alpha\cdot g_2) = \alpha\cdot e(g_1,g_2) $
    \item For generators $g_1\in G_1, \; g_2 \in G_2, \;$ and $g_3 \in G_3,\; e(g_1,g_2) = g_3$     \end{enumerate}    
    \end{definition}
    
    \subsection{Intuition}
    \newcommand{\params}[0]{\textbf{params}}
    The most accessible example of Bilinear Map is in Tripartite Diffie-Hellman Key Exchange. Here the Bilinear Map $e:G_1 \times G_1 \rightarrow G_2$ is defined $e(g_1^a,g_1^b) \rightarrow g_2^{a\cdot b}$ for generators $g_1\in G_1$ and $g_2 \in G_2$. We can think of $a,b$ as secrets and $g_1^a,g_1^b$ as their encodings in $g_1$. Using the map $e$ we can multiply secrets and by multiplying encodings $g^a\cdot g^b = g^{a+b}$ we can also add secrets. 
\par Suppose Alice, Bob and Carol wish to agree upon a single secret key for the three of them. To do so Alice would run an instance generation function that outputs a description of the group $G_1$, a generator $g_1$ and a bilinear map $e$ as we defined above. Then she would choose a secret $a \in \Z_p$ and broadcast $(e \in \params, g_1, pk_a := g_1^a)$. Then Bob would choose $b\in \Z_p$ and broadcast $pk_b := g_1^b$ and Alice would do the same broadcasting $pk_c := g_1^c$. For each party to learn the secret key they would need to run:
\begin{align*}
\text{Alice:} & \; e(pk_b,pk_c)^a = e(g_1^b,g_1^c)^a = ( g_2^{b\cdot c} )^a = g_2^{a\cdot b \cdot c} = sk\\
\text{Bob:} &\; e(pk_a,pk_c)^b = g_2^{a \cdot b \cdot c} = sk \\
\text{Carol:}& \; e(pk_a,pk_b)^c = g_2^{a \cdot b \cdot c} = sk
\end{align*}

\par The important thing to understand is that this bilinear map makes it possible for 3 parties to share a secret without risk of an eavesdropper also being able to know the secret. To generalize, a $k$-multilinear map allows us to encode $k$ secrets easily, so if each party involved in the exchange knows a secret this allows for $k+1$ non interactive key exchange.

    \subsection{Hardness Assumption}
    
    In order to talk about how secure it is to run protocols that use bilinear maps (such as Alice, Bob, and Carol's key exchange) we need to formalize what the adversary would need to be able to do in order to compromise security. We call these formalizations \textit{hardness assumptions}.
    \par First we define the following problems.
    
    \begin{definition}{Bilinear Computational Diffie-Hellman Problem}
    \par Let $G_1,G_2$ be cyclic groups of prime order $p$ and $e:G_1 \times G_1 \to G_2$ be a bilinear map. For $a,b,c,\in \Z_p$ chosen uniformly and $g_1,g_2$ generators of $G_1,G_2$ respectively. The Computational Diffie-Hellman problem is: given $e,g_1,g_1^a,g_1^b$ and ,$g_1^c$; compute $$e(g_1,g_1^{a\cdot b \cdot c}) = g_2^{a\cdot b \cdot c}$$
    \end{definition}
    
    The computational Diffie-Hellman problem might seem a bit confusing, but it is pretty much just a formalization of what we realized after discussing tripartite key exchange: multiplying two secrets is easy using bilinear maps, but that multiplying by a third secret is difficult. Thus we have
    \begin{assumption}{Bilinear CDD hardness assumption}
    \par The bilinear computational Diffie-Hellman problem is hard.
        \end{assumption}
        
        \par Another hardness assumption that we need to make is the bilinear discrete logarithm problem defined below.
       
    \begin{definition}{Bilinear Discrete Logarithm}
    Let $G_1,G_2$ be cyclic groups of prime order $p$ and $e:G_1 \times G_1 \to G_2$ be a bilinear map. For $a \in \Z_p$ chosen uniformly, and generators $g_1,g_2$ of $G_1,G_2$ respectively, the bilinear discrete logarithm problem is: given $g_1,g_1^\alpha, G_1, G_2,$ and $e$ find $\alpha$.
    \end{definition}
    
    Note that while the above problem may seem easy, the encoding $g_1^\alpha = g $ just looks like some element of $G_1$, and since $g_1$ is a generator of $G_1$, \textit{any} element of $G_1$ can be written as $g_1^\beta \; \forall \beta \in \Z_p$. We want it to be difficult to get a secret out of its encoding, so we also rely on the following assumption.
    
    \begin{assumption}{Bilinear DL hardness assumption}
    \par The bilinear discrete logarithm problem is hard
    \end{assumption}
   
    
   \par There are arguments about if these are \textit{good} assumptions to be making, but these arguments are outside the scope of this thesis. It is enough for us that so far cryptographers and mathematicians have been unable to give efficient solutions to solving these problems in general cases.
    
    
    \chapter{Multi-Linear Maps}
    
    \section{Cryptographic Multilinear Maps}
    
    
    \subsection{Dream Definition}
    
    In order to construct indistinguishability obfuscation or functional encryption, we need to be able to do computations in public on encoded secrets in such a way that only information about the final output can be discerned. Computationally, we want to be able to encode, add, multiply and negate secrets. 
    \par In the previous chapter, we saw that bilinear maps allow us to multiply two encoded secrets securely, but if we want to be able to do more complex computations involving multiple multiplications, or sums of products, or products of sums, we need a new cryptographic tool that allows for greater functionality.
    \begin{definition}{Cryptographic Multilinear Map}
    \par Let $G_1,G_2, \cdots, G_k,G_T$ be cyclic groups each of the same prime order $p$ with generators $g_1,g_2,\cdots g_k$ respectively.. Then we say a map $e:G_1 \times G_2 \times \cdots \times G_k \rightarrow G_T$ is a k-multilinear map if
    
    \begin{enumerate}
    \item For $g_i \in G_i \: \forall i \in [K]$ where $[K]$ is the set $\{1,2,\cdots,k \}$, and $\alpha \in \mathbb{Z}_p$ then $e(g_1, g_2, \cdots,g_\ell^\alpha , \cdots , g_k) =e(g_1, g_2, \cdots, g_\ell , \cdots , g_k)^\alpha \\$
    \item if $g_i$ generates $G_i$ then $e(g_1,g_2,\cdots , g_k)$ generates $G_T$. We say that $e$ is \textbf{non-degenerate} if this property holds.
    \end{enumerate}
    \end{definition}
       
       For $g_i \in G_i$ and $\alpha \in \Z_p$, we can think of $g_i^\alpha$ as a public encoding of secret $\alpha$. In this setting, we would use the map $e$ to do $k$ multiplications of encodings in our secure computation. For example for $a,b,c,d,e,f,h \in \Z_p$ and generators $g_1 \in G_1, g_2\in G_2, g_3\in G_3$ and $g_4 \in G_4$we could compute $(a+b)\cdot(c+d)\cdot(e+f+h)$ with the $3$-linear map $e:G_1\times G_2 \times G_3 \to G_4$ with 
 \begin{align*}      
 e(g_1^a \cdot g_1^b,g_2^c\cdot g_2^d,g_3^e\cdot g_3^f \cdot g_3^h)&=e(g_1^{a +b},g_2^{c+d},g_3^{e+f+h})\\
 &= e(g_1,g_2,g_3)^{(a+b)(c+d)(e+f+h)} \\
 &=g_4^{(a+b)(c+d)(e+f+h)}
 \end{align*}
 
 \subsection{Hardness Assumption}
 
 As before we would like to formalize what is being assumed to be difficult in order to be able to discuss the security of using multilinear maps to do computations on secrets in the clear. They are just an extensions of the bilinear discrete logarithm problem and decisional Diffie-Hellman problem.
 
 \begin{definition}{Multilinear Discrete-Log} \par
 The same as for the Bilinear DL problem except an adversary is also given the index $i<k$ to know which group they are working in.
 \end{definition}
 
 
 \begin{definition}{Multilinear Decisional Diffie-Hellman Problem} 
\par Let $\alpha,\alpha_1,\alpha_2,\cdots \alpha_k, \alpha_{k+1} \in \Z_p$ be chosen uniformly. The MDDH problem is: given $G_1,G_2, \cdots, G_k,G_T$ cyclic groups each of the same prime order $p$ with generators $g_1,g_2,\cdots, g_k$ respectively, $g_1^{\alpha_1}, g_2^{\alpha_2}, \cdots, g_i^{\alpha_i} , g_i^{\alpha_{i+1}},\cdots g_k^{\alpha_{k+1}}$ and multilinear map $e$; distinguish 
$$\prod_{j=1}^{k+1}\alpha_j \cdot e(g_1,g_2,\cdots,g_k)  \quad \text{from} \quad \alpha \cdot e(g_1,g_2,\cdots ,g_k)$$
 \end{definition}    
  

\par We assume both of these problems to be hard.

            
    
    \section{Graded Encoding Schemes}
    
      
    \subsection{Intuition}
    To help build an intuition for how Graded Encoding Schemes work, we begin with a simplified definition, that allows us to use familiar syntax, and then we'll build up to the actual definition from there.    
    
    \begin{definition}{k-Graded Encoding Scheme (simplified)} 
    \par Given cyclic groups $G_1,G_2, \cdots , G_k$ of prime order $p$. We call the family of bilinear maps $$e_{i,j}:G_i \times G_j \rightarrow G_{i+j} \text{ for all } 0 \leq i,j \leq k \text{ where } i+j \leq k$$
    a simplified Graded Encoding Scheme. For any secret $\alpha \in \Z_p$ and generator $g_i$ of $G_i$ we call $g_i^\alpha$ a level i encoding of $\alpha$ where elements of $\Z_p$ are called level zero encodings. 
    \end{definition}
    
    In this Encoding Scheme, secrets $\alpha_1,\alpha_2, \cdots \in \Z_p$ are initially encoded as they were with bilinear maps, by taking a generator $g_1\in G_i$ and raising it to the $\alpha_1$ so that it looks like $g_1^{\alpha_1}$. As stated in the definition we call this a level one encoding.
    \par For a level one encoding of elements, computation on secrets is the same as before, but now if we have $e_{1,1}(g_1^{\alpha_1},g_1^{\alpha_2})=g_2^{\alpha_1\cdot \alpha_2}$ for generator $g_2$ of $G_2$ and want to continue to do computations on the new secret $\alpha_1 \cdot \alpha_2$ with another secret $\alpha_3$ we can `multiply' the level one encoding $\alpha_3$ with a level one encoding of 1 to get $e_{1,1}(g_1^{\alpha_3},g_1^1) = g_2^{\alpha_3}$ and continue to do operations as before such as $e_{2,2}(g_2^{\alpha_3},g_2^{\alpha_1\cdot \alpha_2 }) = g_4^{\alpha_3\cdot \alpha_1 \alpha_2}$ for $g_4$ generator of $G_4$. So using this scheme, we can do computations on multiple secrets as long as there are at most $k$ multiplications.
    
    
    \subsection{Encodings}
    As we said in the beginning of this section, we gave a simplified definition of graded encoding schemes in order to give intuition into the \textit{graded} structure of the scheme. In our simplified definition, encodings are unique and of the form $g_i^\alpha$. In practice there are sets of randomized valid encodings for a secret $\alpha \in \Z_p$ at any level, we denote the set of valid encodings of $\alpha$ at level $i: S_i^{(\alpha)}$.
    
    
    \begin{definition}{Level $i$ Encoding}
    \par Let $R$ be a cyclotomic ring, $R_q = R/qR$ for prime $q, \; g\in R_q$ be small, $z \in R_q$ chosen randomly (so it won't be small), and $I = \langle g\rangle$ the principal prime ideal of $g$. We define a valid level $i$ encoding of $\alpha \in R_g \cong \Z_p$ to be anything in the set
    $$S_i^\alpha =\bigg\{\bigg[\frac{a}{z^i} \bigg]_q \bigg| \; a \in \alpha + I \text{ and $a$ is small}\bigg\}$$
    \end{definition}
    For more details into the definition of short, long, and why it is safe to assume $z$ is invertible in this ring, see [\cite{GGH13}], we will instead show how this encoding allows us to achieve multilinear map like functionality. 
    \par Let $a \in S_0^\alpha$ and $b \in S_0^\beta$. Then 
    
    \newcommand{\Encode}[2]{\bigg[ \frac{#2}{z^{#1}}\bigg]_q}
    $$
    \Encode{i}{a} \cdot \Encode{j}{b} = \Encode{i+j}{a \cdot b}
   $$
   So multiplication of a level $i$ encoding with a level $j$ encoding gives a level $i+j$ encoding of their product.
   
   
   $$\Encode{i}{a} + \Encode{i}{b} = \Encode{i}{a+b}$$
   
   Thus this encoding gives us the properties we were looking for in a graded encoding scheme. 
   \newcommand{\encode}[2]{\big[ \frac{#2}{z^{#1}}\big]_q}
   
   
   
   
   \subsection{Definition}
   
   \par Now we are ready for the formal definition of graded encoding schemes, after which we will formalize the procedures that we expect along with such a graded encoding scheme.
         
    \begin{definition}{k-Graded Encoding Scheme}
    \par A k-Graded Encoding System consists of a ring $R$ and a system of sets $\mathcal{S}= \{S_i^{(\alpha)} \subset \{0,1\}*| \alpha \in R, \; 0 \leq i \leq k \}$, such that:
    \begin{enumerate}
    \item For every fixed index i, the sets $\{S_i^{(\alpha)}| \alpha \in R \}$ are disjoint (meaning they form a partition of $S_v := \cup_\alpha S_v^{(\alpha)}$)
    
    \item There is an associative binary operation '$ + $' and a self-inverse unary operation '$-$' such that $\forall \alpha_1,\alpha_2 \in R$, index $i\leq k$, and $u_1 \in S_i^{(\alpha_1)}$ and $u_2 \in S_i^{(\alpha_2)}$, it holds that 
    $$u_1 + u_2 \in S_i^{(\alpha_1 + \alpha_2)} \text{ and } -u_1 \in S_i^{(-\alpha_1)}$$
    where $\alpha_1 + \alpha_2$ and $-\alpha_1$ are addition and negation in $R$.
    
    \item There is an associative binary operation '$\times$' ( on $\{ 0,1 \}*$) such that for every $\alpha_1,\alpha_2 \in R$ every $i_1,i_2$ with $i_1+i_2 \leq k$ and every $u_1 \in S_{i_1}^{(\alpha_1)}$ and $u_2 \in S_{i_2}^{(\alpha_2)}$ it holds that 
    $$u_1 \times u_2 \in S_{i_1 + i_2}^{(\alpha_1 \cdot \alpha_2)} $$
    Where $\alpha_1 \cdot \alpha_2$ is multiplication in $R$, and $i_1 + i_2$ is integer addition.
    \end{enumerate}
    \end{definition}

    
 
  
    
     
         
    
    \subsection{Procedures}
    \begin{itemize}
    \item \textbf{Instance Generation}: 
    	\begin{itemize}
    \item Inputs: $\lambda$, our security parameter, and $k$ the maximum number of multiplication our computation might need (note how this will end up being the degree of our $k$-graded encoding scheme)  
    \item Outputs: $\mathcal{S}$, the description of a $k$ graded encoding scheme as described above, and $p_{zt}$ a level $k$ zero-test parameter which we will explain with some detail in the next section.
    	\end{itemize}  
	\item \textbf{Ring Sampler}: takes $\mathcal{S}$ as input and outputs a level-zero encoding $a \in S_0^{(\alpha)}$ for uniform $\alpha \in R$. The details of how this procedure is constructed is important, but outside the scope of this thesis. What is important to understand is that because this gives an encoding of a uniform \textit{plaintext} element of R without  indicating what that plaintext element is, there is a negligible probability that an adversary would be able to get and recognize a plaintext encoding of 1 or something else that would compromise the security of the multilinear map. However this does give companions of the party that ran the Instance Generation the ability to participate in $k$-nary key exchange by running the Ring Sampler, saving their plaintext encoding, and then publicizing a higher level encoding.
	\item \textbf{Encoding}: takes as input $\mathcal{S}$, the level zero encoding $a\in S_0^{(\alpha)}$, and $i$ and then outputs $u\in S_i^{(\alpha)}$, a level $i$ encoding of $a$
	\item \textbf{Addition, negation, multiplication}: all as we've already described
	%\item \textbf{Zero-test}: takes as inputs $\mathcal{S}$ and $u$, outputs 1 if $u \in S_k^{(0)}$ and 0 otherwise. When used with subtraction this gives us the ability to do equality checking. Note that the zero test only works with level $k$ encodings meaning we can only get information about the encoded elements at the last step of a computation. This is where multilinear maps and graded encoding schemes are different from encrypted computation methods like fully homomorphic encryption. With graded encoding schemes we can only check if the \textit{result} of a computation is what we expect, while with FHE any and all steps of the computation can be decrypted in full even the initial values. This is why graded encoding schemes are a crucial tool for functional encryption where we want this exact property.
	%\item \textbf{Extraction}:?????
    \end{itemize}
    
    
    \subsection{Zero Test}
   \par It is important to note that while randomizing encodings makes the scheme harder to break in many ways, it also makes getting information from encoded values much more difficult. In fact it makes it so much more difficult to get information from encoded values that we can only check to see if it is or isn't zero. In other words, we need a way to determine wether or not $\encode{k}{u}$ is in $0 + I = I = \langle g \rangle$. We call this operation the zero test. To do this we need to define a new \textit{zero test} parameter which we will denote $p_{zt}$.
   
   \begin{definition}{Zero Test}
   \par Let $R$ be a cyclotomic ring, $R_q = R/qR$ for prime $q, \; g, h \in R_q$ be small, $z \in R_q$ chosen randomly (so it won't be small), and $I = \langle g\rangle$ the principal prime ideal of $g$. Define the zero test parameter to be
      
   $$p_{zt} = \bigg[\frac{h \cdot z^k}{g}\bigg]_q $$
   
   Now for arbitrary level $k$ encoding $u_k$ of some unknown plaintext $u$ we define the zero test to be 1 if $p_{zt}\cdot u_k$ is small and 0 if $p_{zt}\cdot u_k$ is large.
   \end{definition}
   
   As before we won't go into the details of small and large, but the designation of small or large in the zero test sense is consistent with our earlier usages of those terms. Below is how we are able to link our earlier usages of the words small and large with the zero test. 
   
   \begin{align*} 
   p_{zt}\cdot u_k  &=  \bigg[\frac{h \cdot z^k}{g}\bigg]_q  \cdot \Encode{k}{u} \text{ since $u_k  \in u + I$ defined to be small $r$ must be small} \\
   &=  \bigg[ \frac{h\cdot (u + r \cdot g ) }{g} \bigg]_q 
   \intertext{If $u_k$ is an encoding of zero then}
   &=\bigg[ \frac{h\cdot (0 + r \cdot g ) }{g} \bigg]_q \\
   &= \big[ h \cdot r \big]_q \intertext{where both $h$ and $r$ are small so $h\cdot r$ is small.}
   \end{align*}
   Otherwise if $u_k$ is not an encoding of zero then
   $$\bigg[ \frac{h\cdot (u + r \cdot g ) }{g} \bigg]_q \approx \bigg[ \frac{h\cdot u }{g} \bigg]_q$$
   Where $h$ and $g$ are small and $u$ is probably not small so $$\bigg[ \frac{h\cdot u }{g} \bigg]_q$$
   is big.

    
    
    
    \subsection{Hardness Assumptions}
    
    The hardness assumptions we have discussed up to this point are pretty standard and well studied, but with the creation of new cryptographic tools comes new hardness assumptions. For this graded encoding scheme a new hardness assumption had to be formulated and while it is inspired by DDH.
        
    \begin{definition}{Graded DDH}
    \begin{align*}
    1. &\; (\mathcal{S},p_{zt}) \leftarrow \text{InstGen}(\lambda,k) && \\
    2. & \text{ For } i = 1,\cdots,k+1: && \\
    3. &\quad \text{ Choose } \alpha_i \leftarrow \text{samp}(\mathcal{S}) && \text{\# get $k+1$ level-0 encodings} \\
    4. &\quad \text{ Set $u_i \leftarrow$ encode$(\mathcal{S},1,\alpha_i)$} &&\text{\# get level-1 encodings of $\alpha_i$} \\
    5. & \text{ Set $\tilde{\alpha} = \prod_{i=1}^{k+1}$}\alpha_i &&\text{\# product of $k+1$ level-0 encodings} \\
    6. & \text{ Choose $\bar{a} \leftarrow$ samp($\mathcal{S}$)} &&\text{\# get level-0 encoding of random element} \\
    7. & \text{ Set $\tilde{u} \leftarrow$ encode$(\mathcal{S},k,\tilde{a})$} &&\text{\# level-$k$ encoding of the product} \\
    8. & \text{ Set $\bar{u}\leftarrow$ encode$(\mathcal{S},k,\bar{a})$} &&\text{\# level-$k$ encoding of random element} \\
    9. & \boldsymbol{u} \leftarrow \{\tilde{u},\bar{u} \} \text{ chosen uniformly} &&
    \end{align*}
    \par Given $\mathcal{S},u_1,u_2,\cdots,u_{k+1},$ and $\boldsymbol{u}$, determine if $\boldsymbol{u}$ is $\tilde{u}$ or $\bar{u}$.
    \end{definition}
    
    As with our other DDH definitions, when we assume this is hard, we are really saying that in a $k$-graded encoding scheme, it is hard to determine the product of $k+1$ level one encodings. We won't go much into how \textit{good} this assumption is, we will say that this encoding scheme is far less studied than, say, RSA and so the cryptanalysis is much less extensive. However there has been some cryptanalysis done on graded encoding schemes [\cite{GGH13}] and overall, it seems that depending on what type of security is needed, different variations on the scheme can be used to protect against different attacks.     
    
    
    
    
    
    \chapter{Indistinguishability Obfuscation}
    
    \section{Circuit Representation}
    While multilinear maps are a central tool used for the obfuscation of circuits, it is difficult to see how when most of us think of circuits as boolean expressions of the form
    $$C(x,y,z) = (x \cdot y) + z $$
    which denotes the circuit ($x$ and $y$) or $z$. A first thought might be to encode $x,y$ and $z$ with a $k$-graded encoding scheme, however this would not hide the operations used in $C$, only the values of the starting parameters. For circuit obfuscation we want it to be hard to determine what $C$ is.
    
    \subsection{Branching Programs}
    
    \par We want to find another way of representing a circuit $C$ that will allow us to obscure its inner workings. To do this we work with variations of a data structure called a branching program. We start with an example of a branching program to help give an intuition, note that $\theta$ and $I$ indicate the output of the program to be 0 and 1 respectively.
      \par For inputs $x_1,x_2,x_3,x_4\in \{ 0 ,1\}$  a branching comparison program for $x_1x_2 > x_3x_4$ is given by:
    \begin{align*}
    i &&\text{ inp}(i) &&\text{if inp}(i)=0 \text{ go to} &&\text{if inp}(i)=1 \text{ go to} \\
    1. &&1 &&2 &&3 \\
    2. &&3 &&4 &&\theta \\
    3. &&3 &&I &&4 \\
    4. &&2 &&\theta &&5 \\
    5. &&4 &&I &&\theta
    \end{align*}
    
    In the above program, the first column $i$ indicates the current line of the program we are at, the second column inp$(i)$ indicates the index of the input to focus on (i.e. inp$(1)$ indicates to focus on $x_1$). The third and fourth columns give us either the next step of the program to go to depending on the value of inp$(i)$; or an output value $\theta$ or $I$.
    \par With this intuition we formalize a variant of branching programs that uses matrices rather than line numbers to get the same functionality.

    
    \begin{definition}{(Oblivious) Branching Program}
    \par  An oblivious branching program of length-$n$ for $\ell$-bit inputs is a sequence
    $$BP = ((\text{inp}(i),A_{i,0},A_{i,1}))_{i=1}^n $$
    Where the $A_{i,b}$'s are permutation matrices in $\{0,1\}^{5 \times 5}$, and inp$(i)\in [\ell]$ is the input bit position examined at step $i$ of the branching program. The function computed by this branching program is
    $$f_{BP,A_0,A_1}(x):= \begin{cases}
    1 \quad \quad \text{if } \prod_{i=1}^n A_{i,x_{inp(i)}} = I \\
    0 \quad \quad \text{otherwise} 
    \end{cases}$$
    \end{definition}
    
     \par As opposed to the example branching program, one of two matrices $A_{i,0}$ or $A_{i,1}$ is chosen based on the bit value of the input indicated by inp$(i)$. Also, note that we can randomize these branching programs in the following way
    
    \begin{definition}{Randomized Branching Programs}
      \newcommand{\inp}[0]{\text{inp}}
     \par Given a branching program $\mathcal{BP} = (\inp(i),A_{i,0},A_{i,1})_{i=1}^n$ and $n$ random $5 \times 5$ invertible matrices $R_1,R_2,\cdots, R_n$. A random branching program is the branching program made up of $\tilde{A_{i,b}} = R_{i-1}\cdot A_{i,b}\cdot R_i^{-1}$ of the form 
      $$\mathcal{RBP} = (\inp(i),\tilde{A_{i,0}},\tilde{A_{i,1}})_{i=1}^n$$
     \end{definition}

\par Not only does randomization help to obscure the underlying circuit, it also prevents any sort of attack that involves reordering the matrices, since matrix multiplication isn't commutative.
      
       
            
      \subsection{Barrington's Theorem}
      
       Originally branching programs were used for oblivious two party computation, but we can start to use them in the context of obfuscating circuits with the following theorem.
      
      
      \begin{theorem}{Barrington's Theorem}
      \par For any depth $d$ fan-in-2 circuit $C$, there exists a branching program consisting of at most $4^d$ permutation matrices $A_{i,b}\in\{0,1\}^{5 \times 5}$ that computes the same function as the circuit $C$.
      \end{theorem}
      
      
     \par Note that in order to keep these branching programs efficient we need our circuit to have logarithmic depth, so this confines us to circuits in Nick's Class (NC$^1$) although later we will talk about generalizing to polynomial sized circuits. 
     
     \section{Approaching Obfuscation for NC$^1$}
     \newcommand{\inp}[0]{\text{inp}}
     
     \subsection{Kilian's Protocol}
     The first attempt of making indistinguishability obfuscation out for randomized branching programs was Kilian's protocol described below. 
     \par Suppose Alice is the obfuscater and Bob is the evaluator. Alice wants to obfuscate circuit $C\in \mathcal{C}$ where $\mathcal{C}$ is the circuit family of all circuits with the same domain and codomain as $C$. Alice starts with a universal circuit $U$ \textbf{say more about this or have a citation ready} that can describe any circuit in $\mathcal{C}$ and $x$ the corresponding input into $U$ such that $U(x,\cdot) = C(\cdot)$. Alice then gets the randomized branching program of 
     $$U = \mathcal{RBP}_U = (\inp(i),\tilde{A_{i,0}},\tilde{A_{i,1}})_{i=1}^n$$
     Then Alice selects all $\tilde{A}_{i,x_{\inp(i)}}$ for all $i$ with $\inp(i)$ corresponding to a bit in $x$. Once this is done Alice sends the partially selected random branching program to Bob who selects matrices corresponding to his input $y$. 
     \par It's worth thinking about why this is a good start for obfuscation. When Bob receives the branching program he has no way of knowing which of the two matrices Alice selected, all he sees of the matrices corresponding to her inputs are $\tilde{A}_i$, just the matrix itself and its order in the product. Also, the randomization prevents Bob from being able to reorder the matrices of the branching program 
     \par Unfortunately, Kilian's protocol alone is broken. However we can further strengthen this scheme using a graded encoding scheme. 
    
    \subsection{Encoded Random Branching Programs}
    
    \par Our first potential fix for Kilian's protocol uses the tool described in chapter 2, graded encoding schemes. As before Alice has a random branching program of a universal circuit, she selects the matrices corresponding to her input $x$ that describes the functionality of $C$; but now she generates a $n$-graded encoding scheme $\mathcal{S}$ with maps $e_{i,j}$, (where $n= 4^d$ and $d$ is the depth of her original circuit $C$) and replaces the entries of each matrix with there level-1 encoding. Now, when Bob multiplies the encoded random branching program together, Alice can send over a level $k$ encoding of the identity matrix so that Bob can zero test to see if the output of his computation is $0$ or $1$ on his input. 
    \par For example, in Killian's Protocol $A_{i,\inp(i)}$ might be of the form
    $$ A_{i,\inp(i)} = 
    \begin{bmatrix}
    1 &0 &0 &1 &0\\
    0 &1 &0 &0 &0\\
    0 &0 &1 &0 &0\\
    0 &0 &0 &0 &0\\
    0 &0 &0 &0 &1\\
    \end{bmatrix}
    $$
    but in the encoded random branching programs construction
    \newcommand{\e}[1]{\leftarrow S_1^{#1}}
    
    
    $$
    A_{i,\inp(i)} = 
    \begin{bmatrix}
    a_{1,1} \e{1} & a_{1,2}\e{0} & a_{1,3}\e{0} & a_{1,4}\e{1} & a_{1,5}\e{0}\\
    a_{2,1} \e{0} & a_{2,2}\e{1} & a_{2,3} \e{0} & a_{2,4} \e{0} &a_{2,5}\e{0}\\
    a_{3,1} \e{0}  &a_{3,2}\e{0}  &a_{3,3}\e{1} &a_{3,4}\e{0}  &a_{3,5}\e{0} \\
    a_{4,1}\e{0} &a_{4,2}\e{0} &a_{4,3}\e{0} &a_{4,4}\e{0} &a_{4,5}\e{0}\\
    a_{5,1}\e{0} &a_{5,2}\e{0} &a_{5,3}\e{0} &a_{5,4}\e{0} &a_{5,5}\e{1}\\
    \end{bmatrix}
    $$
    where each $a_{i,j}$ is chosen uniformly from all small elements of $S_1^{b}$.
    \par At a first glance, this might seem like all we need to obfuscate a circuit. However it doesn't quite hold up against partial evaluation attacks which take advantage of the ordering of circuit representations or mixed input attacks where an adversary might choose matrices $A_{i,\inp(i)}$ inconsistently in the sense that the $j$th bit of the input might be treated as 1 the first time $\inp(i)=j$ and then 0 the next time $\inp(i) = j$. Defenses against both of these attacks rely heavily on randomization techniques involving the adding of ``Bookends'' and ``Multiplicative Bundling'' both of which, while important to constructing $i\mathcal{O}$, are highly technical and don't add much to understanding $i\mathcal{O}$. For our purposes we will acknowledge that our given scheme is not totally complete, but that fixes do exist and are described in detail in [\cite{Garg:2013}].
    
    
   
     
   
    
    \section{Obfuscation for Poly-sized Circuits}
    In this section we use the construction of indistinguishability obfuscation for Nick's Class along with Fully Homomorphic Encryption (FHE) to get obfuscation for polynomial sized circuits. 
    
    \subsection{Fully Homomorphic Encryption}
    We won't get into much detail as to how FHE is constructed, instead we will describe what is expected of a FHE scheme. Like with our previously described forms of encryption, a FHE scheme is a 4-tuple of algorithms containing the standard key generation, encryption, and decryption functions that we know and love. However, with FHE we include another function \textit{evaluate} that takes as input a public key, a function, and a collection of ciphertexts and outputs the encrypted value of the function run on the encrypted input messages. 
    
    \begin{definition}{Fully Homomorphic Encryption}
    \par For message space $\mathcal{M}$, cipher space $C$, key space $K$, and circuit family $\mathcal{F}$ we define a Fully Homomorphic Encryption scheme to be the following functions:
     
     \newcommand{\fhe}[0]{_\text{FHE}}
     
     \begin{itemize}
     \item KeyGen$_\text{FHE}:\Z \to K \times K$ \\
     Defined KeyGen$_\text{FHE}(\lambda)\to (pk,sk)$ where $\lambda$ is the security parameter and $pk,sk$ are a public and secret key respectively 
     \item Enc$_\text{FHE}:K \times \mathcal{M}\to C$ \\
     Defined Enc$_\text{FHE}(pk,m) \to c$ meaning anyone with the public key can encrypt a message and get back a cipher text
     \item Dec$_\text{FHE}:K \times C \to \mathcal{M}$ \\
     Defined Dec$_\text{FHE}(sk,c) \to m$ and it is expected that FHE schemes respect the public key encryption definition of correctness so Dec$_\text{FHE}(sk,$Enc$_\text{FHE}(pk,m))=m$ and where the running time of Dec$_\text{FHE}$ depends only on the security parameter $\lambda$ and not the input.
     \item Eval$_\text{FHE}:K \times \mathcal{F} \times C \times C \times \cdots \times C \to C$ \\
     Defined Eval$\fhe(pk,f,c_1,c_2,\cdots,c_\ell) = c_f$ for $f\in \mathcal{F}$.
    \end{itemize}
    \par We say a FHE scheme is correct if for $c_1 =$Enc$\fhe(pk,m_1), \cdots, c_\ell =$ Enc$\fhe(pk,m_\ell)$,
    $$\text{$\dec(sk,$ Eval$\fhe(pk,f,c_1,c_2, \cdots, c_\ell)) = f(m_1,m_2,\cdots,m_\ell)$ }$$ 
    \end{definition}
    
    \par As opposed to the notion of functional encryption introduced in the first chapter, with FHE decryption is all or nothing just like with classical encryption. A good use for FHE would be if Alice wants to have a large amount of data on a server, and be able to do computations on it. If Alice encrypts her data using a FHE scheme, she can then run Eval commands on the data, download the output of the evaluation command, and decrypt to find it's value. This works well because Alice doesn't need to download all the data onto her computer, decrypt, and then run the computation, but she can still use her data and feel confident that it is staying private.
    \par FHE would not be good for Alice if she wants to give her friend Bob access to a function of the data (like an average or the median) but does not want Bob to be able to see \textit{all} the data. If encrypted using an FHE scheme, giving Bob the secret key would not only allow him to evaluate the average and then decrypt (Dec$_\text{FHE}(sk,$Eval$(pk,\text{average},c_1,\cdots,c_\ell))$, but also decrypt any piece of data he wants (Dec$_\text{FHE}(sk,c)$ for any $c$) thus compromising the security of Alice's data.
    
    
    \subsection{Toward Poly-sized Circuit Obfuscation}
    
    \newcommand{\ionc}[0]{i\mathcal{O}_\text{NC}}
     \newcommand{\fhe}[0]{_\text{FHE}}

    \par For our construction of circuit obfuscation for polynomial sized circuits, we assume that we have an indistinguishability obfuscater for Nick's Class which we will denote $\ionc$, and a fully homomorphic encryption scheme $($Setup$_\text{FHE},$ Encrypt$_\text{FHE},$ Eval$_\text{FHE},$ Decrypt$_\text{FHE})$ where we can assume Decrypt$\fhe$ can be realized by a family of circuits in NC \textbf{FOR MORE INFORMATION SEE (CITATION TO JOSH'S THESIS)}. Last we also assume we have access to $\{ U_\lambda\}$ where each $U_\lambda \in \{ U_\lambda \}$ is a universal circuit polynomial in the size of $\lambda$ and where $U_\lambda(C,m) = C(m)$ for circuit $C\in \{C_\lambda\}$ and message $m$.
    \par Our scheme consists of two functions, Obfuscate and Evaluate. Informally, to obfuscate we will get FHE encryption of our circuit encoding for $C$ which we will call $g$, then to evaluate the encrypted circuit on a message $m$ we run the FHE evaluate function on a universal circuit $U_\lambda$ with $m$ built in to get $U_\lambda(g,m)$ which by correctness of FHE should decrypt to $C(m)$. More formally we construct Obfuscate and Evaluate as follows:
    $\\$
        \par Obfuscate($\lambda, C \in \mathcal{C}_\lambda$) where $\mathcal{C}_\lambda$ are all circuits with size polynomial to $\lambda$.
    \begin{enumerate}
    \item Generate $(PK\fhe,SK\fhe) \leftarrow $ Setup$\fhe(\lambda)$
    \item Get FHE of circuit $C$, $g=$ Encrypt$\fhe(PK\fhe,C)$
    \item Define the program $P(\cdot) = $ Decrypt$\fhe(SK\fhe,\cdot)$, the decryption algorithm with the secret key in it and then obfuscate it $\mathcal{P} = \ionc(P)$, note that we are relying on \textbf{CITATION CITATION} that Decrypt$\fhe\in$ NC
    \item Public Parameters $PP = (\mathcal{P},PK\fhe,g)$
    \end{enumerate}    
    
    \par Evaluate($PP,m$) for valid message $m$.
    \begin{enumerate}
    \item Compute $e=\;$ Eval$\fhe(PK\fhe,U_\lambda(\cdot,m),g) = U_\lambda(g,m) =$ Encrypt$\fhe(PK\fhe,U_\lambda(C,m))$ by correctness of FHE
    \item Output $\mathcal{P}(e)=\ionc($Decrypt$\fhe(SK\fhe, e)) = C(m)$
    \end{enumerate}

    
    \par At first glance the construction above might appear to yield indistinguishability obfuscation for poly-sized circuits, after all we have implicitly assumed our FHE scheme is correct and secure, so the circuit encrypted to get $g$ should be well hidden. Unfortunately the current scheme is not quite secure under our definition of indistinguishability obfuscation. 
    \par Suppose an adversary $\mathcal{A}$ gives us circuits $C_0,C_1 \in \{ C_\lambda\}$ where 
   \begin{center}
    $C_0(X = x_0||x_1) = x_0 \text{ or } x_1$\\
    \text{and} \\
   $C_1(X =x_0||x_1) = x_1 \text{ or } x_0$
    \end{center}
    Where $C_0 = C_1$ since `or' is commutative, so these are valid circuits to test if our scheme meets the indistinguishability obfuscation definition. Now let the encodings of $C_0$ and $C_1$ both be of length $\ell = p(\lambda)$ for some polynomial $p$, and let them be written in the same ordering specified above where the first $k$ bits of the encoding of $C_0$ represents the term ``$x_0$" and likewise the first $k' \approx k$ bits of the encoding of $C_1$ represent the term ``$x_1$". Then if we run Obfuscate$(C_0)\to PP_0$ and Obfuscate$(C_1) \to PP_1$ and give $PP_b$ to the adversary $\mathcal{A}$ for $b \leftarrow \{0,1\}$ chosen uniformly, consider the following attack.
    \par Upon receiving $PP_b =(\mathcal{P}_b,PK\fhe^b,g_b),\; \mathcal{A}$ wants to differentiate $PP_0$ from $PP_1$ by which bit appears first in the encoding of their associated circuits. To do this $\mathcal{A}$ generates a mask $w$ of length $\ell$ such that the first $k$ bits of the mask are all 1 and the last $\ell - k$ bits are all 0 so that it's of the form $w = 11\cdots 100 \cdots 0$. Then $\mathcal{A}$ encrypts the mask $w_\enc =$ Encrypt$\fhe(PK\fhe^b, w)$ and then multiplies the encrypted circuit $g_b$ by the encrypted mask by getting $g_b' =$Eval$\fhe(PK\fhe,\prod,w_\enc,g_b)$ and then for message $m=01$ output $\mathcal{P}_b($Eval$\fhe(PK\fhe,U_\lambda(\cdot,m),g_b')\to b'$. Note that if $b' = 0$ then $x_0$ appears first in the encoding giving away that $\mathcal{A}$ got an obfuscation of $C_0$, and similarly $b'=1$ implies that $\mathcal{A}$ got an obfuscation of $C_1$. Thus this construction doesn't meet our security definition of indistinguishability obfuscation.
       \par To fix this construction we need a way of ensuring that evaluation is run \textit{only} on $g$ without any alterations. In order to do this we need another cryptographic primitive.
       
       \subsection{Low Depth Non Interactive Zero Knowledge Proofs}
       
       Non interactive zero knowledge (hence forth abbreviated NIZK) proofs are a cryptographic primitive between the \textit{prover} and then the \textit{verifier}. We won't give any sort of formal construction of how NIZK proofs are made, rather we focus on what they ensure. Informally, we expect a NIZK proof to work as follows:
       \par Let Peggy be our prover and Victer be our verifier. Given a statement and a witness to that statement a Peggy should be able to generate a proof $\pi$ such that when $\pi$ is given to Victer, he is convinced that $\pi$ is a valid proof and yet anyone given $\pi$ should be unable to use $\pi$ to gain any information about the witness to the statement.
       \par More formally a NIZK proof is a one time interaction between a prover and verifier of sending a proof $\pi$ from the prover to the verifier where $\pi$ has the following properties:
       \begin{itemize}
       \item Perfect Completeness: A proof system is perfectly complete if an honest prover can generate a proof $\pi$ that $x$ exists in language $L$ such that an honest verifier is always convinced that $\pi \implies x\in L$.
       \item Statistical Soundness: A proof system is sound if it is infeasible for a prover to generate a false proof that convinces an honest verifier
       \item Computational Zero Knowledge: A proof system is computational zero-knowledge if the proof does not reveal any information about the witness to any adversary.
       \item Low Depth: We say a proof is low depth if the verifier can be implemented in NC.
         \end{itemize}
    
    
    \par Rough details into how a low depth NIZK proof system could be constructed can be found in appendix B of [\cite{Garg:2013}], for our purposes we only need to know that such proof systems exist and guarantee the above properties.
    
    \subsection{Obfuscation for Poly-Sized Circuits}
    
    \par How might we use low depth NIZK proofs to fix our previous construction of $i\mathcal{O}_p$? As stated at the end of section 3.3.2, we need a way of ensuring that our decryption program only decrypts cipher texts where the message has been \textit{fully} evaluated by our encrypted circuit $g$.
    \par We begin our edits of the previous scheme by first revising how our program $P$ runs. We define $P$ to be a function that not only takes a cipher text $e$ as before, but now also takes a proof $\pi$ and a message $m$ such that for $(PK\fhe,SK\fhe) \leftarrow$ Setup$\fhe(\lambda)$:
    
   
   \par$P(e,m,\pi) =$
   \begin{itemize}
   \item First we check if $\pi$ is a valid low-depth proof that
   $$e =\text{Eval}\fhe(PK\fhe,U_\lambda(\cdot,m),g) $$
   \item If the check fails output $\perp$ indicating the inputs are invalid; otherwise output Decrypt$\fhe(e,SK\fhe)$
   \end{itemize}
    
    \par In essence, the evaluator acts as a prover for the statement ``Did you appropriately generate your cipher text?" and the program $P$ acts as the verifier for proofs $\pi$ of that statement. Here ``appropriately generated" means the entire encrypted circuit $g$ was run on the specified message $m$. Since we know indistinguishability obfuscation is the best possible obfuscation we can assume that any adversary with $\ionc(P)$ can't alter the hard coded $g$ in $P$.
    \par The only other thing we need to check is that our program $P\in$NC so that we can use $\ionc$ on it. As before we know/assume Decrypt$\in$NC, and if we use a low-depth NIZK proof system for step 1 then the verifier$\in$NC so $P\in$NC. Thus we have constructed $i\mathcal{O}_P$ an indistinguishability obfuscater for poly-size circuits.
    
    \chapter{Functional Encryption}
    \section{Definition}
    Now that we have built up an understanding of multilinear maps and circuit obfuscation, we are ready to talk about functional encryption. Recall the following definition of functional encryption where secret keys are generated for different functionalities.
    \begin{definition}{Functional Encryption}
    \par A functional encryption scheme for a class of functions $\mathcal{F} = \mathcal{F}(\lambda)$ over message space $\mathcal{M} = \mathcal{M}_\lambda$ consists of four algorithms $\mathcal{FE} = \{$ Setup, KeyGen, Encrypt, Decrypt$\}$ where:
    \begin{itemize}
    \item Setup$_{FE}(1^\lambda) \to (PP, MSK)$ takes security parameter and outputs Public Parameter and Master Secret Key 
    \item KeyGen$_{FE}(MSK,f)\to SK_f$ takes Master Secret Key and a function $f\in \mathcal{F}$ and outputs a corresponding secret key $SK_f$
    \item Encrypt$_{FE}(PP,m) \to c$ takes Public Parameter and a message $m\in\mathcal{M}$ and outputs cipher text $c$
    \item Decrypt$_{FE}(SK_f,c)\to m'$ takes secret key and cipher text and outputs message $m'$
    \end{itemize}
    \par We say a functional encryption scheme is correct if $$\text{Decrypt$(SK_f$,Encrypt$(PP,m))$ = $f(m)$}$$
    \end{definition}
    
    \section{Construction}
    Here we give a simplified construction of the functional encryption scheme described in the previous section, a more detailed construction using non interactive zero knowledge proofs can be found in \cite{Garg:2013}. Here we assume that we have access to a public key encryption scheme with algorithms (Setup$_{\text{PKE}}$, Encrypt$_{\text{PKE}}$, Decrypt$_{\text{PKE}}$), then
    
    \begin{itemize}
    \item Setup$_{ FE}(1^\lambda)$: Takes security parameter $\lambda$ and generates
     $$(PP, MSK) \leftarrow \text{Setup}_{\text{PKE}}(1^\lambda)$$
     \item KeyGen$_\text{ FE}$(MSK,$f$): Outputs an obfuscation $SK_f := \mathcal{O}(P_f(MSK, \cdot))$ for the program $P_{f,MSK}(\cdot) :=$ Decrypt$_\text{PKE}(MSK,f(\cdot))$
     \item Encrypt$_\text{ FE}$(PP, m$\in \{ 0,1 \}^n$): Outputs $c :=$ Encrypt$_\text{ PKE}(PP, m)$
     \item Decrypt$_\text{ FE}(SK_f,c)$: Outputs $m'$ by running the obfuscated program $SK_f$ on $c$.

    
    \end{itemize}
    \par It might take some rereading of the simplified construction above to understand what's going on. Essentially, our $SK_f$ is an obfuscated program which first fully decrypts a cipher text, then outputs the result of running $f$ on the decrypted message. The idea of fully decrypting the cipher text and then evaluating the function on the decrypted message is why we need obfuscation for this scheme to work since working with an obfuscated program should give no more information than just its input and output behaviors (at-least as we said before this is how indistinguishability obfuscation is often treated).
    
    
    
    
    
    \chapter{ A Brief Introduction to the 5-GenC library}
     \textbf{delete this eventually} \cite{5genc}
     
    \section{ The DSL}
    \section{Circuits and Branching Programs}
    \section{Base and MMaps}
    
    \chapter{Experiments}
    \section{Comparison Circuit}
    \section{Runtime Evaluation}
    
    
    \chapter{Conclusion} 
    
    
    
    

 \bibliographystyle{APA/apa-good}  % or
 \bibliography{thesis}
 % Comment the above two lines and uncomment the next line to use biblatex-chicago.
 %\printbibliography[heading=bibintoc]

% Finally, an index would go here... but it is also optional.
\end{document}
