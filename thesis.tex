% This is the Reed College LaTeX thesis template. Most of the work 
% for the document class was done by Sam Noble (SN), as well as this
% template. Later comments etc. by Ben Salzberg (BTS). Additional
% restructuring and APA support by Jess Youngberg (JY).
% Your comments and suggestions are more than welcome; please email
% them to cus@reed.edu
%
% See http://web.reed.edu/cis/help/latex.html for help. There are a 
% great bunch of help pages there, with notes on
% getting started, bibtex, etc. Go there and read it if you're not
% already familiar with LaTeX.
%
% Any line that starts with a percent symbol is a comment. 
% They won't show up in the document, and are useful for notes 
% to yourself and explaining commands. 
% Commenting also removes a line from the document; 
% very handy for troubleshooting problems. -BTS

% As far as I know, this follows the requirements laid out in 
% the 2002-2003 Senior Handbook. Ask a librarian to check the 
% document before binding. -SN

%%
%% Preamble
%%
% \documentclass{<something>} must begin each LaTeX document
\documentclass[12pt,twoside]{reedthesis}
% Packages are extensions to the basic LaTeX functions. Whatever you
% want to typeset, there is probably a package out there for it.
% Chemistry (chemtex), screenplays, you name it.
% Check out CTAN to see: http://www.ctan.org/
%%
\usepackage{graphicx,latexsym} 
\usepackage{amssymb,amsthm,amsmath}
\usepackage{longtable,booktabs,setspace} 
\usepackage{chemarr} %% Useful for one reaction arrow, useless if you're not a chem major
\usepackage[hyphens]{url}
\usepackage{rotating}
\usepackage{natbib}
% Comment out the natbib line above and uncomment the following two lines to use the new 
% biblatex-chicago style, for Chicago A. Also make some changes at the end where the 
% bibliography is included. 
%\usepackage{biblatex-chicago}
%\bibliography{thesis}

\newtheorem{definition}{Definition}
% \usepackage{times} % other fonts are available like times, bookman, charter, palatino

\title{Multi-Input Functional Encryption and Obfuscation}
\author{Sage R. Michaels}
% The month and year that you submit your FINAL draft TO THE LIBRARY (May or December)
\date{May 2018}
\division{Mathematics and Natural Sciences}
\advisor{Dylan McNamee}
%If you have two advisors for some reason, you can use the following
%\altadvisor{Your Other Advisor}
%%% Remember to use the correct department!
\department{Mathematics}
% if you're writing a thesis in an interdisciplinary major,
% uncomment the line below and change the text as appropriate.
% check the Senior Handbook if unsure.
\thedivisionof{The Established Interdisciplinary Committee for}
% if you want the approval page to say "Approved for the Committee",
% uncomment the next line
%\approvedforthe{Committee}

\setlength{\parskip}{0pt}
%%
%% End Preamble
%%
%% The fun begins:
\begin{document}

  \maketitle
  \frontmatter % this stuff will be roman-numbered
  \pagestyle{empty} % this removes page numbers from the frontmatter

% Acknowledgements (Acceptable American spelling) are optional
% So are Acknowledgments (proper English spelling)
    \chapter*{Acknowledgements}
	I want to thank a few people.

% The preface is optional
% To remove it, comment it out or delete it.
    \chapter*{Abstract}
	This is an example of a thesis setup to use the reed thesis document class.
	

    \tableofcontents
% if you want a list of tables, optional
%    \listoftables
% if you want a list of figures, also optional
 %   \listoffigures
    
 
  \mainmatter % here the regular arabic numbering starts
  \pagestyle{fancyplain} % turns page numbering back on   
    
     \chapter*{Introduction}
         \addcontentsline{toc}{chapter}{Introduction}
	\chaptermark{Introduction}
	\markboth{Introduction}{Introduction}
	        
    
    \chapter{Background}
    \section{Encryption}
    In plain English, Encryption a way to share a message so that only the intended recipient(s) of
    that message are able to read it. Historically this was done by means of obscurity, in the sense that correspondents
    assumed only they knew the specific method by which messages between them would be encrypted. The problem
    with Encryption by obscurity is that as soon as a method of obscurity becomes popular, it immediately becomes
    obsolete.
    \subsection{Classical Encryption}
    Now, Cryptographers work to develop encryption schemes that are computationally infeasible for adversaries to break even if the method of encryption is known (this is known as Kerckhoff's Principle). To do this, Encryption functions are made public but take an extra parameter that is kept secret, we call this secret a key, and the best keys are ones that are chosen randomly, since they are nearly impossible to guess. In defining an encryption scheme it is important to note that there exists a keyspace $K$ which is the set of all valid keys, a message space $M$ made up of all valid messages, and a cipher space $C$ the set of all valid cipher-texts (encryptions of messages).
 \\ \\
  We define an encryption scheme $\Pi$ to be the following three functions:
     
 
 $$\text{Gen}:\mathbb{Z} \rightarrow K \times K$$
 Defined to be for $\lambda \in \mathbb{Z}$, Gen($\lambda ) \rightarrow (pk,sk)$ where $pk$ and $sk$ are seemingly random keys of length $\lambda$.
 
  $$\text{Enc}:K \times M \rightarrow C$$
Defined to be for key $pk\in K$ and message $m\in M$ Enc$_pk(m) \rightarrow c$ for some cipher text $c\in C$
 
 $$\text{Dec}:K \times C \rightarrow M$$
 Defined to be for key $sk \in K$ and cipher text $c\in C$ Dec$_sk(c) \rightarrow m$ for some message $m\in M$
\par It is important to note that if $sk = pk$ this is called a symmetric or private key encryption scheme meaning only the correspondents know the key and they keep it secret. If $sk \not= pk$ then this is called an asymmetric or public key encryption scheme where $sk$ is a secret key and $pk$ is a public key. In a public key encryption scheme anyone can encrypt a message since the public key is public, but only people with the secret key are able to decrypt.


\begin{definition}[Correctness]
In this setting we say an encryption scheme $\Pi$ is \textbf{correct} if for $n\in \mathbb{Z} , (sk,pk) \leftarrow$ Gen$(n)$ and $m\in M$ 

$$\text{Dec}_{sk}(\text{Enc}_{pk}(m)) = m$$
\end{definition}


\par Suppose Alice and wants to send Bob a secret message $m$. To do this Bob would have to run Gen$(n) \rightarrow pk, sk$ and then send $pk$ to Alice. Then Alice gets $c:= $Enc$_{pk}(m)$ and sends $c$ over to Bob. Finally Bob gets $m':= $Dec$_{sk}(c)$. If the scheme is correct then $m' = m$ and Bob is able to read Alice's message. The above interaction is represented in the following diagram.
\begin{center}
\textbf{insert sick diagram of Alice interacting with Bob}
\end{center}


\newcommand{\Enc}[0]{Enc}
\newcommand{\Dec}[0]{Dec}
\subsection{Functional Encryption}

With classical encryption, decryption is all or nothing, either you have the secret key and can find out the message, or you don't have the secret key so you can't. With functional encryption we broaden the possibilities of what is communicated between senders in an encryption scheme. We start with a definition and then show the formal construction.
\begin{definition}[Correctness]
A Functional Encryption Scheme $\Pi$ is \textbf{correct} if for $m \in M$, some predetermined function $f$ with $M$ as it's domain, and appropriate $(pk,ek)\in K$ generated by $\Pi$'s key generation algorithm:

$$\Dec_{ek}(\Enc_{pk}(m)) = f(m) $$
\end{definition}

It's easy to see that this definition encapsulates the older definition of correctness by making $f$ the identity function $f(m) = m$, but this syntax covers many other cryptographic primitives as well like Attribute Based Encryption and Identity Based Encryption. To see how these primitives are sub cases of Functional Encryption. Lets formalize our idea of a Functional Encryption Scheme. 
\par To define a Functional Encryption Scheme, we must first define a way of describing what a cipher text can be decrypted to.

\begin{center}
\textbf{Think of something better than case space, it's confusing with the notation for a cipher space. The paper calls it a key space K but that's also confusing. Change later, keep in mind now.}
\end{center}


\begin{definition}[Functionality]
Given a case space $C_ase \cup \{\epsilon\}$, message space $M$ we define the functionality $F$ to be 
$$F:C_ase \times M \rightarrow M$$
\end{definition}

\par Functionality describes what the possible outputs are. In public key encryption, knowing the secret key $sk$ allows for the message to be read in full, but without the secret key, only the length of the message can be discerned from the cipher text. To write this in the syntax of a functionality we define
$$
F(c,m) = 
\begin{cases}
 x \text{ if } c = 1 \\
\text{length}(x) \text{ if } c = \epsilon
\end{cases}
$$
The only functionality of public key encryption is fully decoding the message so this is our primary case $(c = 1)$, however we also account for the information learned without the public key which is an unavoidable rather than built in case $(c = \epsilon)$.


\newcommand{\Z}[0]{\mathbb{Z}}
\begin{definition}[Functional Encryption Scheme]
A Functional Encryption Scheme $\Pi$ is defined to be the following algorithms:

$$\text{setup}: \Z \rightarrow K \times K$$
Defined: For $\lambda \in \Z,$ setup($\lambda) \rightarrow (pk,mk)$, generates a public key and master key
$$\text{Gen}: K \times C \rightarrow K $$
Defined: For $c\in C_{ase}, mk \in K$, Gen$(mk,c) \rightarrow sk$ which is kept secret and is the secret key of functionality $c$.

$$\Enc: K \times M \rightarrow C_{ipher}$$
Defined: For $pk\in K$ and $m \in M, \Enc_{pk}(m) \rightarrow c$

$$\Dec:K \times C_{ipher} \rightarrow M$$
Defined: For $ek \in K$ and $c \in C_{ipher}, \Dec(k,c) \rightarrow n$ where $n = F(k,m)$ for some functionality $F$. 
\end{definition}


If the notion of a Functionality was confusing before, the use of it in generating the secret key should make it clear.
\par Functional Encryption is at the early stages of development now, but is an extremely powerful tool. From Functional Encryption we can easily describe variations like Attribute Based Encryption, Identity Based Encryption, and Multi Input Functional Encryption.


    \section{Black Box Obfuscation}
    Later on we will get into the details of the current notion of Obfuscation, known as Indistinguishability
    Obfuscation. However, to understand its importance and it's limitations it's important to understand
    where it evolved from. 
    \section{Diffie-Hellman Key Exchange}
    
    \chapter{Multi-Linear Maps}
    \section{Definition}
    \section{Intuition}
    \section{Construction Outline}
    \section{Candidate Goups/Quotient Rings/Fields}
    
    
    
    \chapter{Indistinguishability Obfuscation}
    \section{Definition}
    \section{Construction}
    \section{Usage, Limitations, and Goals}
    
    
    
    
    \chapter{Multi-Party Input Functional Encryption}
    \section{Scheme}
    \section{Construction}
    \section{Limitations and Goals}
    
    
    
    \chapter{ A Brief Introduction to the 5-GenC library}
    \section{ The DSL}
    \section{Circuits and Branching Programs}
    \section{Base and MMaps}
    
    \chapter{Experiments}
    \section{Comparison Circuit}
    \section{Runtime Evaluation}
    
    
    \chapter{Conclusion} 
    
    
    
    

 \bibliographystyle{APA/apa-good}  % or
 \bibliography{thesis}
 % Comment the above two lines and uncomment the next line to use biblatex-chicago.
 %\printbibliography[heading=bibintoc]

% Finally, an index would go here... but it is also optional.
\end{document}
